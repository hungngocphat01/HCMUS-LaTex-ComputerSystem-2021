%%%%%%%%%%%%%%%%%%%%%%%%%%%%%%%%%%%%%%%%%
% University Assignment Title Page 
% LaTeX Template
% Version 1.0 (27/12/12)
%
% This template has been downloaded from:
% http://www.LaTeXTemplates.com
%
% Original author:
% WikiBooks (http://en.wikibooks.org/wiki/LaTeX/Title_Creation)
%
% License:
% CC BY-NC-SA 3.0 (http://creativecommons.org/licenses/by-nc-sa/3.0/)
%
%%%%%%%%%%%%%%%%%%%%%%%%%%%%%%%%%%%%%%%%%

\title{Ôn tập cuối kỳ môn Kỹ thuật lập trình}
%%%%%%%%%%%%%%%%%%%%%% PACKAGE INCLUSIONS %%%%%%%%%%%%%%%%%%%%%% 
\documentclass[12pt]{report}
\usepackage{extsizes}
\usepackage[T5]{fontenc}
\usepackage[dvipsnames]{xcolor}
\usepackage[utf8]{inputenc}
\usepackage{csquotes}
\usepackage[vietnamese,english]{babel}
\usepackage{amsmath}
\usepackage[outputdir=build,cache=false]{minted}
% Xoá đoạn "[outputdir=build,cache=false]" ở dòng trên nếu compile trên Overleaf
\usepackage{float}
\usepackage{graphicx}
\usepackage[colorinlistoftodos]{todonotes}
\usepackage{listings}
\usepackage[unicode]{hyperref}
\usepackage{enumitem}
\usepackage{fancyhdr}
\usepackage{subfiles}
\usepackage{forest}
\usetikzlibrary{arrows.meta,
                chains,
                positioning,
                quotes,
                shapes.geometric}

\usepackage{background}
\usepackage{geometry}

%%%%%%%%%%%%%%%%%%%%%% DOCUMENT FORMATTING %%%%%%%%%%%%%%%%%%%%%% 
\geometry{
    a4paper,
    total={170mm,250mm},
    left=20mm,
    top=30mm,
 }
\hypersetup{
    colorlinks=true,
    linkcolor=blue,
    filecolor=magenta,      
    urlcolor=blue,
    citecolor=blue
}

\pagestyle{fancy}
\fancyhf{}
\rhead{Lớp 19CTT4}
\lhead{Tài liệu ôn tập Hệ thống máy tính - Hợp ngữ 8086}
\rfoot{Trang \thepage}

\setlength{\parindent}{0pt}
\setlength{\parskip}{0.3em}
\setlength{\headheight}{15pt}

\AtBeginEnvironment{minted}{
    \renewcommand{\fcolorbox}[4][]{#4}}       

\tikzstyle{arrow} = [thick,->,>=stealth]
\tikzstyle{startstop} = [rectangle, rounded corners, minimum width=3cm, minimum height=1cm,text centered, draw=black, fill=red!30]
\tikzstyle{io} = [trapezium, trapezium left angle=70, trapezium right angle=110, minimum width=3cm, minimum height=1cm, text centered, draw=black, fill=blue!30]
\tikzstyle{other_process} = [rectangle, minimum width=3cm, minimum height=1cm, text centered, draw=black]
\tikzstyle{process} = [rectangle, minimum width=3cm, minimum height=1cm, text centered, draw=black, fill=orange!30]
\tikzstyle{decision} = [diamond, minimum width=3cm, minimum height=1cm, text centered, draw=black, fill=green!30]
%%%%%%%%%%%%%%%%%%%%%% MACRO DEFINITIONS %%%%%%%%%%%%%%%%%%%%%% 
\newcommand{\nocontentsline}[3]{}
\newcommand{\tocless}[2]{\bgroup\let\addcontentsline=\nocontentsline#1{#2}\egroup}

\newcommand{\cd}[1]{\texttt{#1}}

\renewenvironment{figure*}
{\begin{figure}[H]\center}
{\end{figure}}

\newcommand{\cdh}[1]{\textcolor{red}{\cd{#1}}}

%%%%%%%%%%%%%%%%%%%%%%%%%%%%%%%%%

\begin{document}

\begin{titlepage}
    \vspace*{\fill}

    \centering
    \textsc{\LARGE Đại học Khoa học Tự nhiên}\\[0.5cm]
    \textsc{\large ĐẠI HỌC QUỐC GIA TP. HCM}\\[0.5cm]
    
    {\Large Lớp 19CTT4}\\[1.5cm]

    \rule{\textwidth}{0.4pt} \\[0.4cm]
    {
        \huge \bfseries Tài liệu ôn thi Hệ thống máy tính\\
        Nội dung: Hợp ngữ 8086
    }
    \rule{\textwidth}{0.4pt}\\[1.5cm]
    
    {\Large Học kỳ 2, 2020 -- 2021}\\[1.5cm]
    {\large Đảm nhiệm soạn thảo: Hùng Ngọc Phát \\
    Powered by \LaTeXe}
    \vspace*{\fill}

\end{titlepage}

%%%%%%%%%%%%%%%%%%%%%%%%%%%%%%%%%

\section*{Lời nói đầu}
Tài liệu này rất dài vì bản thân cái ngôn ngữ này đã là a pain in the ass rồi. Vì có tới 4 lớp học 4 chương trình khác nhau nên tài liệu này không thể bao trùm hết được, mong các bạn thông cảm. Ngoài ra nếu bạn nào trên lớp không học chủ đề này thì có thể bỏ qua hẳn chương này luôn, hoặc có thể đọc để biết lớp 19\_2 học khổ như thế nào \cd{:((((((((((}.\bigskip

Vì phần lý thuyết dài nên đọc chán lắm, các bạn nên đọc phần \nameref{chapterBaiTap} trước rồi nếu có gì thắc mắc thì đọc phần \nameref{chapterLyThuyet} sau. \bigskip

Tài liệu này \textbf{không} nói về hợp ngữ x86 32-bit sử dụng Windows API mà chỉ nói về hợp ngữ của vi xử lý 8086 (16-bit) sử dụng DOS API trên máy tính IBM PC. Để biết thêm thông tin về x86 32-bit, vui lòng tham khảo slide của lớp các bạn. Tuy nhiên, vì sườn chung của hợp ngữ là giống nhau nên các bạn cũng có thể tham khảo tài liệu này.\bigskip

À những cái chữ nào màu xanh là hyperlink và có thể bấm vào để nhảy tới phần tương ứng nha.\bigskip

Chúc các bạn thi tốt và qua được cái môn {\tiny củ lìn} này.\\[1.5cm]

Đây là một dự án mã nguồn mở nên mã nguồn {\LaTeX} của tài liệu này được công khai trên github: \href{https://github.com/hungngocphat01/LaTex-ComputerSystem-2021}{hungngocphat01/LaTex-ComputerSystem-2021}.
\pagebreak

\renewcommand*\contentsname{Mục lục}
\setcounter{tocdepth}{2}
\tableofcontents
\pagebreak

\chapter{Lý thuyết} \label{chapterLyThuyet}
\pagebreak

\subfile{1_HTThanhGhi_CTChuongTrinh}
\subfile{2_TruyCapBN_Bien}
\subfile{3_Ngat_NXCoBan}
\subfile{4_CacPhepToan}
\subfile{5_CauTrucDK}
\subfile{6_CongThucChuyenCTDK}
\subfile{7_Stack_ThuTuc}
\subfile{8_ThuVienEMU8086}


\chapter{Một số chương trình cơ bản bằng Assembly 8086} \label{chapterBaiTap}
\subfile{9_BaiTap}


\renewcommand{\bibname}{Tài liệu tham khảo}
\nocite{htmt_lvl}
\nocite{jbwyatt}
\bibliographystyle{IEEEtran}
\bibliography{references}
\end{document}
