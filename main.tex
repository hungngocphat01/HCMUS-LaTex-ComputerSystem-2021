%%%%%%%%%%%%%%%%%%%%%%%%%%%%%%%%%%%%%%%%%
% University Assignment Title Page 
% LaTeX Template
% Version 1.0 (27/12/12)
%
% This template has been downloaded from:
% http://www.LaTeXTemplates.com
%
% Original author:
% WikiBooks (http://en.wikibooks.org/wiki/LaTeX/Title_Creation)
%
% License:
% CC BY-NC-SA 3.0 (http://creativecommons.org/licenses/by-nc-sa/3.0/)
%
%%%%%%%%%%%%%%%%%%%%%%%%%%%%%%%%%%%%%%%%%

\title{Ôn tập cuối kỳ môn Kỹ thuật lập trình}
%%%%%%%%%%%%%%%%%%%%%% PACKAGE INCLUSIONS %%%%%%%%%%%%%%%%%%%%%% 
\documentclass[12pt]{report}
\usepackage{extsizes}
\usepackage[T5]{fontenc}
\usepackage[dvipsnames]{xcolor}
\usepackage[utf8]{inputenc}
\usepackage{csquotes}
\usepackage[vietnamese,english]{babel}
\usepackage{amsmath}
\usepackage[outputdir=build,cache=false]{minted}
% Xoá đoạn "[outputdir=build,cache=false]" ở dòng trên nếu compile trên Overleaf
\usepackage{float}
\usepackage{graphicx}
\usepackage[colorinlistoftodos]{todonotes}
\usepackage{listings}
\usepackage[unicode]{hyperref}
\usepackage{enumitem}
\usepackage{fancyhdr}
\usepackage{subfiles}
\usepackage{forest}
\usetikzlibrary{arrows.meta,
                chains,
                positioning,
                quotes,
                shapes.geometric}

\usepackage{background}
\usepackage{geometry}

%%%%%%%%%%%%%%%%%%%%%% DOCUMENT FORMATTING %%%%%%%%%%%%%%%%%%%%%% 
\geometry{
    a4paper,
    total={170mm,250mm},
    left=20mm,
    top=30mm,
 }
\hypersetup{
    colorlinks=true,
    linkcolor=blue,
    filecolor=magenta,      
    urlcolor=blue,
    citecolor=blue
}

\pagestyle{fancy}
\fancyhf{}
\rhead{Lớp 19CTT4}
\lhead{Tài liệu ôn tập Hệ thống máy tính - Hợp ngữ 8086}
\rfoot{Trang \thepage}

\setlength{\parindent}{0pt}
\setlength{\parskip}{0.3em}
\setlength{\headheight}{15pt}

\AtBeginEnvironment{minted}{
    \renewcommand{\fcolorbox}[4][]{#4}}       

\tikzstyle{arrow} = [thick,->,>=stealth]
\tikzstyle{startstop} = [rectangle, rounded corners, minimum width=3cm, minimum height=1cm,text centered, draw=black, fill=red!30]
\tikzstyle{io} = [trapezium, trapezium left angle=70, trapezium right angle=110, minimum width=3cm, minimum height=1cm, text centered, draw=black, fill=blue!30]
\tikzstyle{other_process} = [rectangle, minimum width=3cm, minimum height=1cm, text centered, draw=black]
\tikzstyle{process} = [rectangle, minimum width=3cm, minimum height=1cm, text centered, draw=black, fill=orange!30]
\tikzstyle{decision} = [diamond, minimum width=3cm, minimum height=1cm, text centered, draw=black, fill=green!30]
%%%%%%%%%%%%%%%%%%%%%% MACRO DEFINITIONS %%%%%%%%%%%%%%%%%%%%%% 
\newcommand{\nocontentsline}[3]{}
\newcommand{\tocless}[2]{\bgroup\let\addcontentsline=\nocontentsline#1{#2}\egroup}

\newcommand{\cd}[1]{\texttt{#1}}

\renewenvironment{figure*}
{\begin{figure}[H]\center}
{\end{figure}}

\newcommand{\cdh}[1]{\textcolor{red}{\cd{#1}}}

%%%%%%%%%%%%%%%%%%%%%%%%%%%%%%%%%

\begin{document}

\begin{titlepage}
    \vspace*{\fill}

    \centering
    \textsc{\LARGE Đại học Khoa học Tự nhiên}\\[0.5cm]
    \textsc{\large ĐẠI HỌC QUỐC GIA TP. HCM}\\[0.5cm]
    
    {\Large Lớp 19CTT4}\\[1.5cm]

    \rule{\textwidth}{0.4pt} \\[0.4cm]
    {
        \huge \bfseries Tài liệu ôn thi Hệ thống máy tính\\
        Nội dung: Hợp ngữ 8086
    }
    \rule{\textwidth}{0.4pt}\\[1.5cm]
    
    {\Large Học kỳ 2, 2020 -- 2021}\\[1.5cm]
    {\large Đảm nhiệm soạn thảo: Hùng Ngọc Phát \\
    Powered by \LaTeXe}
    \vspace*{\fill}

\end{titlepage}

%%%%%%%%%%%%%%%%%%%%%%%%%%%%%%%%%

\section*{Lời nói đầu}
Tài liệu \textit{Ôn thi cuối kỳ Hệ thống máy tính - Hợp ngữ 8086} này được soạn vào học kỳ 2, năm học 2020 - 2021 (07/2021). Vì mỗi bạn học mỗi lớp khác nhau nên chương trình học có lớp kiểu này có lớp kiểu kia (vd như lớp 2 không học \cd{MIPS} mà tất cả đều code \cd{x86}). Trong tài liệu này, mình sẽ cố gắng tóm tắt lại \textbf{tất cả} nội dung của hợp ngữ 8086 \textit{cơ bản} theo sườn nội dung được cung cấp bởi \cd{jbwyatt}, người viết ra trình giả lập \cd{emu8086} và slide của lớp \cd{19\_3} (soạn bởi thầy Lê Viết Long). Bạn nào không học phần nào trên lớp có thể bỏ qua, không cần đọc phần đấy.\bigskip

Ngoài ra, bởi vì phần lý thuyết rất là dài nên các bạn nên đọc phần \ref{chapterBaiTap} \cd{Bài tập} trước rồi nếu không hiểu vì sao nó lại như vậy thì đọc phần \ref{chapterLyThuyet} \cd{Lý thuyết} sau. \bigskip

Chúc các bạn thi tốt và qua được cái môn {\tiny củ lìn} này.\\
\vspace*{\fill}

{\LaTeX} source code: \href{https://github.com/hungngocphat01/LaTex-ComputerSystem-2021}{github: hungngocphat01/LaTex-ComputerSystem-2021}.
\pagebreak

\renewcommand*\contentsname{Mục lục}
\setcounter{tocdepth}{2}
\tableofcontents
\pagebreak

\chapter{Lý thuyết} \label{chapterLyThuyet}
\pagebreak

\subfile{1_HTThanhGhi_CTChuongTrinh}
\subfile{2_TruyCapBN_Bien}
\subfile{3_Ngat_NXCoBan}
\subfile{4_CacPhepToan}
\subfile{5_CauTrucDK}
\subfile{6_CongThucChuyenCTDK}
\subfile{7_Stack_ThuTuc}

\chapter{Một số chương trình cơ bản bằng Assembly 8086} \label{chapterBaiTap}
\subfile{8_BaiTap}


\renewcommand{\bibname}{Tài liệu tham khảo}
\bibliographystyle{IEEEtran}
\bibliography{references}
\end{document}

%%%%%%%%%%%%%%%%%%%%%%%%%%%%%%%%%%%%%%%%%%%%%%%%%%%%
% Comments can be added to the margins of the document using the \todo{Here's a comment in the margin!} todo command, as shown in the example on the right. You can also add inline comments too:

% \todo[inline, color=green!40]{This is an inline comment.}



% \subsection{Tables and Figures}

% Use the table and tabular commands for basic tables --- see Table~\ref{tab:widgets}, for example. You can upload a figure (JPEG, PNG or PDF) using the files menu. To include it in your document, use the includegraphics command as in the code for Figure~\ref{fig:frog} below.

% % % Commands to include a figure:
% % \begin{figure}
% % \centering
% % \includegraphics[width=0.5\textwidth]{frog.jpg}
% % \caption{\label{fig:frog}This is a figure caption.}
% % \end{figure}

% % \begin{table}
% % \centering
% % \begin{tabular}{l|r}
% % Item & Quantity \\\hline
% % Widgets & 42 \\
% % Gadgets & 13
% % \end{tabular}
% % \caption{\label{tab:widgets}An example table.}
% % \end{table}

% \subsection{Mathematics}

% \LaTeX{} is great at typesetting mathematics. Let $X_1, X_2, \ldots, X_n$ be a sequence of independent and identically distributed random variables with $\text{E}[X_i] = \mu$ and $\text{Var}[X_i] = \sigma^2 < \infty$, and let
% $$S_n = \frac{X_1 + X_2 + \cdots + X_n}{n}
%       = \frac{1}{n}\sum_{i}^{n} X_i$$
% denote their mean. Then as $n$ approaches infinity, the random variables $\sqrt{n}(S_n - \mu)$ converge in distribution to a normal $\mathcal{N}(0, \sigma^2)$.

% \subsection{Lists}

% You can make lists with automatic numbering \dots

% \begin{enumerate}
% \item Like this,
% \item and like this.
% \end{enumerate}
% \dots or bullet points \dots
% \begin{itemize}
% \item Like this,
% \item and like this.
% \end{itemize}

% We hope you find write\LaTeX\ useful, and please let us know if you have any feedback using the help menu above.

