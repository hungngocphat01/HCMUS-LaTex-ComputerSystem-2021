\documentclass[main.tex]{subfiles}
\begin{document}

\section{Stack}
Stack là một vùng nhớ nằm trên RAM, được xác định bởi địa chỉ chứa trong thanh ghi SS.\\
Đỉnh của stack được xác định bởi địa chỉ được lưu trong thanh ghi \cd{SP} là \cd{SS:[SP]}.\\
Stack trong CPU 8086 được lưu theo theo chiều ``ngược'', nghĩa là khi ta push vào thì thanh ghi SP sẽ ``đi xuống''.\\
Trong các ngôn ngữ bậc cao, stack segment được sử dụng để chứa các biến cục bộ và tham số của các thủ tục khi được gọi.\bigskip

Để đẩy một phần tử vào stack, ta dùng lệnh \cd{PUSH}, để lấy một phần tử từ stack ra, ta dùng lệnh \cd{POP}. 2 lệnh này chỉ hoạt động với toán hạng có giá trị 2 byte (16 bit).\\
Cú pháp:
\begin{verbatim}
push/pop SREG 
push/pop REG 
push/pop memory 
\end{verbatim}

Ví dụ: swap giá trị của 2 thanh ghi 
\begin{minted}[]{asm}
mov ax, 1234h
mov bx, 0abch

push ax         ; stack <- 1234h 
mov ax, bx      ; ax = 0abch
pop bx          ; stack -> bx 
                ; bx = 1234h
\end{minted}

\section{Thủ tục}
Khái niệm thủ tục trong Assembly giống với khái niệm hàm (void) trong C.\\
Cách khai báo một thủ tục:
\begin{minted}[]{asm}
<tên thủ tục> proc 
    <nội dung thủ tục>
endp
\end{minted}

Vd: khai báo thủ tục hoán đổi 2 thanh ghi \cd{ax} và \cd{bx}
\begin{minted}[]{asm}
swapAXBX proc 
    push ax 
    mov ax, bx
    pop bx
endp
\end{minted}

Để gọi một thủ tục, ta sử dụng lệnh \cd{CALL}.
\begin{minted}[]{asm}
mov ax, 1234h
mov bx, 0abch
call swapAXBX 
\end{minted}

Cách một thủ tục được ``gọi'': sử dụng call stack.
\begin{itemize}
    \item Địa chỉ của thanh ghi \cd{IP} được cất vào stack. (IP là thanh ghi trỏ đến lệnh đang được thực thi).
    \item \cd{IP} sẽ trỏ đến lệnh đầu tiên thủ tục và bắt đầu thực thi các lệnh trong thủ tục.
    \item Sau khi thủ tục được thực thi xong, giá trị của thanh ghi \cd{IP} sẽ được pop ra từ stack để chương trình tiếp tục thực thi tại nơi mà nó đã rời đi trước đó.
\end{itemize}
Đây cũng là cách mà ``callstack'' hoạt động trong các ngôn ngữ bậc cao như C.\\
Vì lý do trên, \textbf{ta tuyệt đối không được hàm hỏng stack} vì sau khi gọi hàm, chương trình sẽ bị trở về sai vị trí.

\end{document}