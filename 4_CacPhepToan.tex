\documentclass[main.tex]{subfiles}
\begin{document}
\section{Các phép toán số học và logic} \label{sec:pheptoan}
\subsection{Phép cộng} \label{subsec:phepcong}
\begin{minted}[]{asm}
add dest, src
\end{minted}
Để thực hiện cộng 2 giá trị, ta sử dụng lệnh \cd{ADD}. Lệnh này nhận vào 2 tham số \cd{dest} và \cd{src} và thực hiện phép gán: \cd{dest += src}.\\
\textbf{Chú ý:} 2 toán hạng không được phép là 2 ô nhớ và thanh ghi đoạn.
\begin{minted}[]{asm}
mov ax, 12d 
add ax, 8d  ; ax += 8
            ; ax = 12 + 8 = 20 
\end{minted}
Lệnh này tác động đến các cờ \cd{AF, CF, OF, PF, SF, ZF}.

\subsection{Phép trừ}
Lệnh: \cd{SUB}. Cú pháp và giới hạn tương tự lệnh \cd{ADD}.
\begin{minted}[]{asm}
mov ax, 12d 
mov bx, 2d
sub ax, bx  ; ax -= 2
            ; ax = 12 - 2 = 10 
\end{minted}
Lệnh này tác động đến các cờ \cd{AF, CF, OF, PF, SF, ZF}.

\subsection{Phép nhân}
Cú pháp: 
\begin{minted}[]{asm}
mul X
\end{minted}
Lệnh \cd{MUL} nhân 2 số không dấu.
Có 2 trường hợp xảy ra:
\begin{itemize}
    \item Nếu \cd{X} có kích thước 8 bit (1 byte), lệnh này sẽ thực hiện phép gán: \cd{AX = AL * X}.
    \item Nếu \cd{X} có kích thước 16 bit (2 byte), lệnh này sẽ thực hiện phép gán: \cd{DXAX = AX * X}.
    \par trong đó DXAX là một thanh ghi tổ hợp (tưởng tượng) gồm high là DX và low là AX.
\end{itemize} 
Lệnh này chỉ tác động đến các cờ \cd{CF, OF}.


\begin{minted}[breaklines]{asm}
; Demo TH1 
mov al, 5d      ; AL, BL là thanh ghi 8 bit
mov bl, 2d 
mul bl          ; AX = AL * BL = 5 * 2 = 10
; Demo TH2 
mov ax, 5d 
mov bx, 2d 
mul bx          ; DXAX = AX * BX = 5d * 2d = 0010d
                ; Lúc này DX = 0 và AX = 10 vì 10 < FFFFh vẫn đủ nhỏ để nằm trong AX
; Demo TH2 (tt)
mov ax, 0DEADh
mov bx, 0DADh
mul bx          ; DXAX = AX * BX = DEADh * DADh = 0BE543E9h
                ; lúc này DX = 0BE5h và AX = 43E9h vì số trên quá lớn để nằm vừa trong AX
\end{minted}

Ngoài ra ta còn lệnh \cd{IMUL} với cú pháp tương tự, nhưng nó sẽ treat 2 toán hạng là 2 số có dấu.

\subsection{Phép chia} \label{subsec:phepchia}
\begin{minted}[]{asm}
div X
\end{minted}
Lệnh \cd{DIV} chia 2 số không dấu.
Có 2 trường hợp xảy ra:
\begin{itemize}
    \item Nếu \cd{X} có kích thước 8 bit (1 byte), lệnh này sẽ thực hiện phép gán: \cd{AL = AX div X} (thương) và \cd{AH = AX mod X} (dư).  
    \par Biểu diễn dưới dạng chia Euler: \cd{AX = X*AL + AH}.
    \item Nếu \cd{X} có kích thước 16 bit (2 byte), lệnh này sẽ thực hiện phép gán: \cd{AX = DXAX div X} và \cd{DX = DXAX mod X}.
    \par Biểu diễn dưới dạng chia Euler: \cd{DXAX = X*AX + DX}.
\end{itemize} 
\begin{minted}[]{asm}
; Demo TH1 
mov ax, 120d
mov bl, 10d ; BL là thanh ghi 8 bit 
div bl      ; AL = thương, AH = dư 
            ; AL = 12, AH = 0
; Demo TH1 (tt)
mov ax, 123d 
mov bl, 2d 
div bl      ; AL = thương = 61 
            ; AH = dư = 1
; Demo TH2 
mov dx, 0
mov ax, DEADh   ; ta phải gán DX = 0 để DXAX = AX 
mov bx, BEEFh 
                ; Thực hiện chia DEADh/BEEFh
div bx          ; AX = thương = 1d
                ; DX = dư = 8126d
; Demo TH2 (tt)
mov dx, 0FACh
mov ax, EB00h   ; DXAX = FACEB00h
mov bx, DEADh 
                ; Thực hiện chia FACEB00h/DEADh
div bx          ; AX = thương = 1205h
                ; DX = dư = 679Fh

\end{minted}
Lệnh này không tác động đến cờ.

\subsection{Lệnh \cd{INC} và \cd{DEC}}
Cú pháp:
\begin{minted}[]{asm}
inc X
dec X
\end{minted}
Tương đương với 2 dòng C sau:
\begin{minted}[]{asm}
X++
X--
\end{minted}
Vd:
\begin{minted}[]{asm}
mov ax, 12d 
inc ax          ; ax = 13d
mov bx, 70d     
dec bx          ; bx = 69d
\end{minted}
2 lệnh này thay đổi cờ tương tự như \cd{ADD} và \cd{SUB}.

\subsection{Các phép toán logic}
\subsubsection{Lệnh \cd{AND}}
\begin{minted}[]{asm}
and dest, src
\end{minted}
Lệnh \cd{AND} nhận vào 2 tham số như trên và thực hiện phép gán: \cd{dest = dest \& src}.\\
Vd:
\begin{minted}[]{asm}
mov ah, 10110110b
mov al, 01101010b
and ah, al          ; AH = AH and AL 
; KQ: AH = 00100010b
\end{minted}
\begin{verbatim}
    10110110b
  & 01101010b
    ---------
    00100010b
\end{verbatim}

\subsubsection{Lệnh \cd{OR}, \cd{XOR}}
Tương tự lệnh \cd{AND}.
\begin{minted}[]{asm}
mov ah, 10110110b
mov al, 01101010b
or ah, al          ; AH = AH or AL 
; KQ: AH = 11111110b
\end{minted}

\subsubsection{Lệnh \cd{NOT}}
\begin{minted}[]{asm}
not X
\end{minted}
Lệnh \cd{NOT} nhận vào 1 tham số như trên và thực hiện phép gán: \cd{X = not X} (đảo các bit của \cd X).
Vd:
\begin{minted}[]{asm}
mov ah, 10110110b
not ah
; KQ:   01001001b
\end{minted}

\subsubsection{Lệnh \cd{NEG}}
Cú pháp:
\begin{minted}[]{asm}
neg X
\end{minted}
Lệnh này thực hiện đảo dấu của X bằng cách thực hiện lấy số bù 2.
Vd:
\begin{minted}[]{asm}
mov ax, 01101100b
neg ax 
; ax = 10010100b
\end{minted}

\subsubsection{Các lệnh dịch bit}
Có 2 lệnh dịch bit là \cd{SHL} (shift left) và \cd{SHR} (shift right). Chúng hoạt động giống toán tử \verb#<<# và \verb#>># trong C và có syntax tương tự nhau:
\begin{minted}[]{asm}
shl X
shr X
\end{minted}
2 lệnh trên để dịch \cd X qua trái (phải) 1 bit. Kết quả dịch sẽ được gán lại trực tiếp vào thanh ghi \cd X, và bit bị ``rơi ra'' sẽ nằm trong thanh ghi cờ \cd{CF}.\\
Vd:
\begin{minted}[]{asm}
mov ah, 10010001b
shl ah 
; AH << 1 => AH = 00100010b, CF = 1

mov ah, 10011100b
shr ah 
; AH >> 1 => AH = 01001110b, CF = 0
\end{minted}
Ngoài ra ta còn có thể dịch nhiều bit cùng lúc:
\begin{minted}[]{asm}
shl X, Y 
shr X, Y
\end{minted}
Mã C tương ứng:
\begin{minted}[]{asm}
X << Y 
X >> Y
\end{minted}
\end{document}