\documentclass[main.tex]{subfiles}
\begin{document}
\section{Ngắt (interrupt)}
Các interrupt giống như những ``thư viện chuẩn'' mà hệ điều hành/BIOS cung cấp cho chúng ta. \\
Mỗi interrupt được đánh một số thứ tự nhất định (vd: \cd{21h}).\\
Bên trong các interrupt là các hàm hệ thống (syscall).\\
Khi ta muốn gọi một hàm hệ thống nào đó, ta lưu mã số của hàm đó trong thanh ghi \cd{AH}, và các thanh ghi còn lại dùng để chứa tham số của hàm đó (thanh ghi nào chứa giá trị gì thì tuỳ vào từng hàm). Cuối cùng ta gọi lệnh \cd{int <mã số interrupt>}.\\
Vd:
\begin{minted}[]{asm}
; In một chuỗi được trỏ tới bởi DX ra màn hình 
; Interrupt 21h, hàm số 09h
mov ah, 09h 
lea dx, str 
int 21h
\end{minted} 
\textit{Khi gọi các hàm được lưu trong interrupt, giá trị của các thanh ghi có thể sẽ bị thay đổi!!}
Danh sách các interrupt phổ biến: \href{https://jbwyatt.com/253/emu/8086_bios_and_dos_interrupts.html}{\cd{8086 BIOS and DOS interrupts}}.

\section{Nhập xuất dữ liệu} \label{sec:nhapxuatdl}
Để nhập xuất dữ liệu trong Assembly, ta không thể làm thủ công mà ta phải nhờ đến các hàm hệ thống có sẵn trong hệ điều hành hoặc BIOS. Dưới đây là một số cú pháp nhập/xuất cơ bản sử dụng API của MS-DOS (\cd{int 21h}).

\subsection{Đọc một kí tự từ bàn phím}
\begin{itemize}
    \item Mã hàm: \cd{01h}.
    \item Tham số: không.
    \item Trả về: \cd{AL = mã ASCII của kí tự vừa đọc}. \cd{AL = 0} nếu đó là kí tự điều khiển.
\end{itemize}
\begin{minted}[]{asm}
mov ah, 01h 
int 21h
\end{minted}

\subsection{In một kí tự ra màn hình}
\begin{itemize}
    \item Mã hàm: \cd{02h}.
    \item Tham số: \cd{DL = mã ASCII của kí tự cần hiển thị}.
    \item Trả về: Không.
\end{itemize}
\begin{minted}[]{asm}
; In kí tự '$' ra màn hình 
mov ah, 02h 
mov dl, '$'     ; hoặc mov dl, 24h (mã ascii của kí tự '$')
int 21h
\end{minted}

\subsection{Hiện chuỗi kí tự ra màn hình} \label{subsec:xuatchuoi}
\begin{itemize}
    \item Mã hàm: \cd{09h}.
    \item Tham số: \cd{DX = địa chỉ offset của chuỗi}.
    \item Trả về: Không.
    \item Lưu ý: chuỗi cần hiển thị phải kết thúc bằng dấu \cd{'\$'}. Dấu \cd{'\$'} được sử dụng để đánh dấu kết thúc của chuỗi, tương tự như ký tự \cd{NULL} trong C.
\end{itemize}
\begin{minted}[]{asm}
str db "Hello world$"
... 
mov ah, 09h 
mov dx, offset str   ; hoặc lea dx, str
int 21h
\end{minted}
\subsection{Đọc chuỗi kí tự từ bàn phím} \label{subsec:docchuoi}
\begin{itemize}
    \item Mã hàm: \cd{0Ah}.
    \item Tham số: \cd{DX = địa chỉ offset của chuỗi}.
    \item Trả về: \cd{[DX+1]}: số ký tự đã đọc. \cd{[DX+2]}: chuỗi đã đọc được.
    \item Lưu ý: buffer chuỗi này phải có cấu trúc như sau:
    \begin{verbatim}
<số ký tự tối đa đọc được>, <số ký tự đã đọc>, <phần nội dung đã đọc được>
    \end{verbatim}
    \par Hàm này không tự chèn dấu \cd{'\$'} vào cuối chuỗi.
\end{itemize}
\begin{minted}[breaklines]{asm}
str db 100, 0, 100 dup('$')
...
mov ah, 0ah 
mov dx, offset str 
int 21h
\end{minted}
Giải thích:
\begin{itemize}
    \item Cấu trúc của biến \cd{str} trong khai báo(từ trái qua phải):
    \begin{itemize}
        \item \cd{db}: mỗi phần tử của chuỗi là 1 byte (hiển nhiên).
        \item \cd{100}: báo cho hệ điều hành biết được nó có thể đọc tối đa 100 kí tự (giống như số \cd 100 trong \cd{getline(cin, str, 100)} trong C++).
        \item \cd{0}: số kí tự mà hệ điều hành đã đọc được. Hiện tại là \cd 0. Sau khi hệ điều hành đọc xong nó sẽ gán giá trị tương ứng cho ô nhớ này.
        \item \cd{100 dup('\$')}: chuỗi ``thực sự'' mà hệ điều hành sẽ lưu dữ liệu đã đọc được vào đây. Ta khởi tạo nó với tất cả phần tử là dấu \verb#'$'#. Giống như khai báo \verb#char str[100] = {'$'}#;
    \end{itemize}
    \item Giả sử ta đã nhập vào chuỗi \cd{"Dm assembly kho vcl"}. Chuỗi này gồm có 18 ký tự. Biến \cd{str} sau khi nhập xong sẽ có giá trị là:
    \begin{verbatim}
100, 18, "Dm assembly kho vcl$$$$$$$$$$$$$$$$..."
    \end{verbatim} 
    \par Sau chuỗi này là các dấu  \verb#'$'#, vì khi khai báo ta đã khai báo một chuỗi có 100 phần tử, nhưng chuỗi đọc được chỉ có 18 phần tử mà thôi. Do đó, 82 phần tử còn lại sẽ là \verb#'$'#.
    \par Ta nên khai báo mảng gồm 100 dấu \verb#'$'# cho tiện in ra màn hình, vì quy định của hàm in ra màn hình là chuỗi phải kết thúc bằng dấu \verb#'$'#, nhưng hàm đọc chuỗi lại không tự thêm dấu \verb#'$'# vào cuối chuỗi sau khi đọc.
\end{itemize}

\subsection*{Trở về DOS}
Sau khi chương trình kết thúc, nó sẽ không tự thoát ra như trong các ngôn ngữ lập trình/hệ điều hành khác. Ta phải tự tay thoát chương trình bằng hàm số \cd{4ch} thuộc interrupt \cd{21h}.
\begin{minted}[]{asm}
mov ah, 4ch
int 21h
\end{minted}

Các ngắt chuẩn của MS-DOS cũng như BIOS của các máy IBM PC chạy vi xử lý 8086 không có các hàm đọc hay in giá trị số. Chúng chỉ có thể đọc và in kí tự hoặc chuỗi kí tự. Lập trình viên phải tự viết các hàm phục vụ những điều này.

\end{document}